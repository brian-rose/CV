%%%%%%%%%%%%%%%%%%%%%%%%%%%%%%%%%%%%%%%%%
% Compact Academic CV
% LaTeX Template
% Version 1.0 (10/6/2012)
%
% This template has been downloaded from:
% http://www.LaTeXTemplates.com
%
% Original author:
% Dario Taraborelli (http://nitens.org/taraborelli/home)
%
% License:
% CC BY-NC-SA 3.0 (http://creativecommons.org/licenses/by-nc-sa/3.0/)
%
% Important:
% This template needs to be compiled using XeLaTeX
%
% Note: this template has the option to use the Hoefler Text font, see the
% font configurations section below for instructions on using this font
%
%%%%%%%%%%%%%%%%%%%%%%%%%%%%%%%%%%%%%%%%%

%----------------------------------------------------------------------------------------
%	PACKAGES AND OTHER DOCUMENT CONFIGURATIONS
%----------------------------------------------------------------------------------------

\documentclass[11pt, letterpaper]{article} % Document font size and paper size

\usepackage{fontspec} % Allows the use of OpenType fonts

\usepackage{geometry} % Allows the configuration of document margins
\geometry{letterpaper, textwidth=5.5in, textheight=8.5in, marginparsep=7pt, marginparwidth=.6in} % Document margin settings
\setlength\parindent{0in} % Remove paragraph indentation

\usepackage[usenames,dvipsnames]{xcolor} % Custom colors

\usepackage{sectsty} % Allows changing the font options for sections in a document
\usepackage[normalem]{ulem} % Custom underlining
\usepackage{xunicode} % Allows generation of unicode characters from accented glyphs
\defaultfontfeatures{Mapping=tex-text} % Converts LaTeX specials (``quotes'' --- dashes etc.) to unicode

\usepackage{marginnote} % For margin years
\newcommand{\years}[1]{\marginnote{\scriptsize #1}} % New command for including margin years
\renewcommand*{\raggedleftmarginnote}{}
\setlength{\marginparsep}{7pt} % Slightly increase the distance of the margin years from the content
\reversemarginpar

\usepackage[xetex, bookmarks, colorlinks, breaklinks, pdftitle={Brian E. J. Rose - vita},pdfauthor={Brian E. J. Rose}]{hyperref} % PDF setup - set your name and the title of the document to be incorporated into the final PDF file meta-information
\hypersetup{linkcolor=blue,citecolor=blue,filecolor=black,urlcolor=MidnightBlue} % Link colors

\newcommand{\publink}{http://www.atmos.albany.edu/facstaff/brose/resources/Publications/}

%. Make a nice header
\usepackage{fancyhdr}
\pagestyle{fancy}
%\lhead{Brian E. J. Rose}
\chead{Brian E. J. Rose}
%\rhead{page \thepage}
\rhead{\thepage}
%\cfoot{center of the footer!}
\cfoot{\scriptsize Last updated: \today}
\renewcommand{\headrulewidth}{0.4pt}
%\renewcommand{\footrulewidth}{0.4pt}

%----------------------------------------------------------------------------------------
%	FONT CONFIGURATIONS
%----------------------------------------------------------------------------------------

% Two font choices are available in this template, the default is Linux Libertine, available for free at: http://www.linuxlibertine.org while the secondary choice is Hoefler Text which comes bundled with Mac OSX.
% To use Hoefler Text, comment out the Linux Libertine block below and uncomment the Hoefler Text block. You will also need to replace the "\&" characters with "\amper{}" in section titles.

% Linux Libertine Font (default)
%\setromanfont [Ligatures={Common}, Numbers={OldStyle}, Variant=01]{Linux Libertine O} % Main text font
%%\setmonofont[Scale=0.8]{Monaco} % Set mono font (e.g. phone numbers)
%\sectionfont{\mdseries\upshape\Large} % Set font options for sections
%\subsectionfont{\mdseries\scshape\normalsize} % Set font options for subsections
%\subsubsectionfont{\mdseries\upshape\large} % Set font options for subsubsections
%\chardef\&="E050 % Custom ampersand character

% Hoefler Text Font (bundled with Mac OSX)
\setromanfont [Ligatures={Common}, Numbers={OldStyle}]{Hoefler Text} % Main text font
\setmonofont[Scale=0.8]{Monaco} % Set mono font (e.g. phone numbers)
\setsansfont[Scale=0.9]{Optima Regular} % Set sans font, used in the main name and titles in the document
\newcommand{\amper}{{\fontspec[Scale=.95]{Hoefler Text}\selectfont\itshape\&}} % Custom ampersand character
\sectionfont{\sffamily\mdseries\large\underline} % Set font options for sections
\subsectionfont{\rmfamily\mdseries\scshape\normalsize} % Set font options for subsections
\subsubsectionfont{\rmfamily\bfseries\upshape\normalsize} % Set font options for subsubsections

%----------------------------------------------------------------------------------------

\begin{document}

%. No header on first page
\thispagestyle{empty}

%----------------------------------------------------------------------------------------
%	CONTACT AND GENERAL INFORMATION SECTION
%----------------------------------------------------------------------------------------

{\LARGE Brian E. J. Rose}\\[0.3 cm] % Your name
Associate Professor \\
Department of Atmospheric \& Environmental Sciences \\
University at Albany (SUNY) \\
ES 351, 1400 Washington Ave., Albany NY \texttt{12222}, U.S.A.\\[.2cm]
Phone: \texttt{(518) 442-4477}\\ % Your phone number
Email: \href{mailto:brose@albany.edu}{brose@albany.edu}\\ % Your email address
%\textsc{url}: \href{http://www.atmos.albany.edu/facstaff/brose/}{http://www.atmos.albany.edu/facstaff/brose/}\\ % Your academic/personal website
Web: \href{http://www.atmos.albany.edu/facstaff/brose/}{http://www.atmos.albany.edu/facstaff/brose/}\\ % Your academic/personal website

Canadian citizen, lawful permanent resident of the USA, fluent in English and French


%\vfill % Whitespace between contact information and specific CV information

%\section*{CURRENT POSITION}

%\emph{Assistant Professor}, Department of Atmospheric and Environmental Sciences, University at Albany (SUNY) % Your current or previous employment position

%------------------------------------------------

%\section*{Areas of specialization}

%Physics; Relativity Theory. % Your primary areas of research interest

%----------------------------------------------------------------------------------------
%	EDUCATION SECTION
%----------------------------------------------------------------------------------------

\section*{EDUCATION}

\years{2010}\textsc{PhD}, Climate Physics and Chemistry, Massachusetts Institute of Technology \\
Oceanic control of the sea ice edge and multiple equilibria in the climate system (Advisor:
  J. Marshall. Awarded 2010 Rossby Prize.)

\years{2002}\textsc{MSc}, Atmospheric \& Oceanic Sciences, McGill University \\
A diagnostic scheme for global precipitation based on
  vertical motion (Advisor: C.A. Lin)

\years{1999} \textsc{BSc}, Atmospheric \& Oceanic Sciences, McGill
  University \\
  Numerical simulation of a mesoscale vortex over the Beaufort Sea (Advisor: M.K. Yau)


\section*{ACADEMIC EMPLOYMENT}

\years{2019 -- } 
  Associate Professor (with tenure), Atmospheric \& Environmental Sciences, University at Albany (SUNY)

\years{2013 -- 2019}
  Assistant Professor, Atmospheric \& Environmental Sciences, University at Albany (SUNY)
  
\years{2012 -- 2013}  Research Associate, Atmospheric Sciences, University of Washington

\years{2010 -- 2012}
  NOAA Climate and Global Change Postdoctoral Fellow, Atmospheric
  Sciences, University of Washington. Host: David S. Battisti

\years{2010}  Postdoctoral Associate, Earth, Atmospheric and Planetary Sciences, MIT

\years{2005 -- 2010}  Research Assistant, Earth, Atmospheric and Planetary Sciences, MIT

\years{2003 -- 2004} Research Assistant, Atmospheric and Oceanic Sciences, McGill
  University

\years{2000} Research Assistant, McGill University and \emph{Centre de recherche en
  calcul appliqu\'{e}}, Montreal


\section*{PUBLICATIONS}\label{publications}

* \emph{indicates student co-author.}

\emph{Reprints available at links below or from \url{http://www.atmos.albany.edu/facstaff/brose/}}

\subsection*{Peer-reviewed publications with UAlbany affiliation}  

\years{2021}
Lin, Y.-J.*, Y.-T. Hwang, J. Lu, F. Liu, and \underline{B.E.J. Rose} (2021), The dominant contribution of Southern Ocean heat uptake to time-evolving radiative feedback in CESM. Geophys. Res. Lett. 48, doi:10.1029/2021GL093302
\vspace{0.2 cm}

\href{\publink Cardinale_etal_JClim2021.pdf}{Cardinale, C.*, \underline{B.E.J. Rose}, A.L. Lang, and A. Donohoe (2021), Stratospheric and Tropospheric Flux Contributions to the Polar Cap Energy Budgets. J. Climate 34, 4261--4278, doi:10.1175/JCLI-D-20-0722.1}
\vspace{0.2 cm}

\href{\publink Henry_etal_JClim2021.pdf}{Henry, M., T.M. Merlis, N.J. Lutsko, and \underline{B.E.J. Rose} (2021), Decomposing the Drivers of Polar Amplification with a Single Column Model. J. Climate 34, 2355--2365, doi:10.1175/JCLI-D-20-0178.1}
\vspace{0.2 cm}

\years{2020} 
Dai, A., D. Huang, \underline{B.E.J. Rose}, J. Zhu and X. Tian (2020), Improved Methods for Estimating Equilibrium Climate Sensitivity from Transient Warming Simulations. Climate Dynamics 54, 4515--4543, doi:10.1007/s00382-020-05242-1
\vspace{0.2 cm}

\href{\publink Rencurrel_Rose_JClim2020.pdf}{Rencurrel, M.C.* and \underline{B.E.J. Rose} (2020), The Efficiency of the Hadley Cell Response to Wide Variations in Ocean Heat Transport. J. Climate 33, 1643--1658, doi:10.1175/JCLI-D-19-0334.1}
\vspace{0.2 cm}

\years{2018} 
\href{\publink Rencurrel_Rose_jcli-d-17-0856.1.pdf}{Rencurrel, M.C.* and \underline{B.E.J. Rose} (2018), Exploring the climatic response to wide variations in ocean heat transport on an aquaplanet. J. Climate 31, 6299--6318, doi:10.1175/JCLI-D-17-0856.1} 
\vspace{0.2 cm}

\href{\publink Rose_JOSS2018.pdf}{\underline{Rose, B.E.J.} (2018), CLIMLAB: a Python toolkit for interactive, process-oriented climate modeling. J. Open Source Software, 3(24), 659, doi:10.21105/joss.00659}
\vspace{0.2 cm}

\years{2017} 
\href{http://advances.sciencemag.org/content/3/11/e1600983}{Hoffman, P.F., D.S. Abbot, Y. Ashkenazy, D.I. Benn, J.J. Brocks, P.A. Cohen, G.M. Cox, J.R. Creveling, Y. Donnadieu, D.H. Erwin, I.J. Fairchild, D. Ferreira, J.C. Goodman, G.P. Halverson, M.F. Jansen, G. Le Hir, G.D. Love, F.A. Macdonald, A.C. Maloof, C.A. Partin, G. Ramstein, \underline{B.E.J. Rose}, C.V. Rose, P.M. Sadler, E. Tziperman, A. Voigt, and S.G. Warren (2017), Snowball Earth climate dynamics and Cryogenian geology--geobiology. Science Advances 3:e1600983, doi:10.1126/sciadv.1600983}
\vspace{0.2 cm}

\href{\publink Singh_et_al-2017-Geophysical_Research_Letters-2.pdf}{Singh, H.A., P.J. Rasch and \underline{B.E.J. Rose} (2017), Increased Ocean Heat Convergence into the High Latitudes with CO$_2$-Doubling Enhances Polar-Amplified Warming. Geophys. Res. Lett. 44, doi:10.1002/2017GL074561}
\vspace{0.2 cm}

\href{\publink Rose_2017_ApJ_846_28.pdf}{\underline{Rose, B.E.J.}, T.W. Cronin and C.M. Bitz (2017), Ice Caps and Ice Belts: the effects of obliquity on ice-albedo feedback. Astrophys. J. 846, doi:10.3847/1538-4357/aa8306}
\vspace{0.2 cm}

\href{\publink Haugstad_et_al-2017-Geophysical_Research_Letters.pdf}{Haugstad, A.D.*, K.C. Armour, D.S. Battisti and \underline{B.E.J. Rose} (2017), Relative roles of surface temperature and climate forcing patterns in the inconstancy of radiative feedbacks. Geophys. Res. Lett. 44, doi:10.1002/2017GL074372}
\vspace{0.2 cm}

\years{2016}
\href{\publink Voigt_et_al-2016-TRACMIP.pdf}{Voigt, A., M. Biasutti, J. Scheff, J. Bader, S. Bordoni, F. Codron, R.D. Dixon, J. Jonas, S.M. Kang, N.P. Klingaman, R. Leung, J. Lu, B. Mapes, E.A. Maroon, S. McDermid, J. Park, R. Roehrig, \underline{B.E.J. Rose}, G.L. Russell, J. Seo, T. Toniazzo, H. Wei, M. Yoshimori, and L.R.V. Zeppetello (2016), The Tropical Rain belts with an Annual Cycle and Continent Model Intercomparison Project: TRACMIP. J. Adv. Model. Earth Syst. 8, 1868--1891, doi:10.1002/2016MS000748}
\vspace{0.2 cm}
 
\href{\publink Rose_Rayborn_CCCR2016.pdf}{\underline{Rose, B.E.J.} and L. Rayborn* (2016), The effects of ocean heat uptake on transient climate sensitivity. Current Climate Change Reports 2, 190--201, doi:10.1007/s40641-016-0048-4}
\vspace{0.2 cm}
   
\href{\publink Rose_Rencurrel_JClim2016.pdf}{\underline{Rose, B.E.J.} and M.C. Rencurrel* (2016), The vertical structure of tropospheric water vapor: comparing radiative and ocean-driven climate changes. J. Climate 29, 4251--4268.}
\vspace{0.2 cm}
 
\years{2015}  
\href{\publink Rose_JGR2015.pdf}{\underline{Rose, B.E.J.} (2015), Stable ``Waterbelt'' climates controlled by tropical ocean heat transport: a non-linear coupled climate mechanism of relevance to Snowball Earth. J. Geophys. Res. 150, doi:10.1002/2014JD022659}
\vspace{0.2 cm}
 
\years{2014}  
\href{\publink Rose_etal_GRL2014.pdf}{\underline{Rose, B.E.J.}, K. Armour, D.S. Battisti, N. Feldl and D. Koll (2014), The dependence of transient climate sensitivity and radiative feedbacks on the spatial pattern of ocean heat uptake. Geophys. Res. Lett. 41, doi:10.1002/2013GL058955}
  
\subsection*{Peer-reviewed publications prior to UAlbany affiliation} 
 
\years{2013}  
\href{\publink Rose_etal_transitions_JClim2013.pdf}{\underline{Rose, B.E.J.}, D. Ferreira and J. Marshall (2013), The role of oceans and sea ice in abrupt transitions between multiple climate states. J. Climate 26, 2862-2879}
\vspace{0.2 cm}

\href{\publink Rose_Ferreira_JClim2013.pdf}{\underline{Rose, B.E.J.} and D. Ferreira (2013), Ocean heat transport and water vapor greenhouse in a warm equable climate: a new look at the low gradient paradox. J. Climate 26, 2117-2136}
\vspace{0.2 cm}

\years{2011}  
\href{\publink FMR_JClim2011_MultipleEq.pdf}{Ferreira, D., J. Marshall and \underline{B.E.J. Rose} (2011): Climate determinism revisited: multiple equilibria in a complex climate model. J. Climate. 24, 992-1012}
\vspace{0.2 cm}
  
\years{2009}  
\href{\publink Rose_Marshall_JAS2009.pdf}{\underline{Rose, B.E.J.} and J. Marshall (2009): Ocean heat transport, sea ice, and multiple climate states: insights from energy balance models. J. Atmos. Sci. 66, 2828-2843}
\vspace{0.2 cm}
  
\years{2003}  
\href{\publink Rose_Lin_2003.pdf}{\underline{Rose, B.E.J.} and C.A. Lin (2003): Precipitation from vertical motion: a statistical diagnostic scheme. Int. J. Climatol. 23, 903-919}


\subsection*{Other publications}
\years{2010}  
\href{\publink Rose_PhD_2010.pdf}{\underline{Rose, B.E.J.} (2010): Oceanic control of the sea ice edge and multiple equilibria in the climate system, PhD thesis, MIT, Cambridge MA}
\vspace{0.2 cm}

\years{2002}  
\underline{Rose, B.E.J.} (2002): A diagnostic scheme for global precipitation based on vertical motion, MSc thesis, McGill University, Montreal.


\section*{WORK IN PROGRESS}\label{work-in-progress}

Min, L.*, D.R. Fitzjarrald, Y. Du, \underline{B.E. J. Rose}, J. Hong, and Q. Min (2021), Exploring sources of bias in HRRR using New York State Mesonet (revised for J. Geophys. Res. Atmospheres)
\vspace{0.2 cm}

Cardinale, C.* and \underline{B.E.J. Rose} (2021), Arctic surface warming efficiency of tropospheric energy flux events. (in prep)
\vspace{0.2 cm} 

Rencurrel, M.C.*, \underline{B.E.J. Rose}, M. Rugenstein, K. Armour (2021), Effects of Spatial Patterns of Ocean Heat Uptake on the inter-model spread of the Transient Climate Response. (in prep.)
\vspace{0.2 cm}

\underline{Rose, B.E.J.} (2021), Climate in the absence of ocean heat transport. (to be submitted to Geophys. Res. Lett.)
\vspace{0.2 cm}

Paiewonsky, P.*, O. Elison Timm and \underline{B.E.J. Rose} (2021), State dependency of the forest-tundra-short wave feedback: comparing the mid-Pliocene and pre-industrial eras using a newly-developed vegetation model. (to be submitted to Climate Dynamics)
\vspace{0.2 cm}

\underline{Rose, B.E.J.}, L. Rayborn* and N. Feldl (2021), Understanding the Dependence of Radiative Feedbacks and Clouds on the Spatial Structure of Ocean Heat Uptake. (in prep.)

\section*{OTHER SCHOLARLY ACTIVITY}\label{other-scholarly-activity}

\subsection*{Developer of Open-Source Scientific Software and Educational Resources}\label{computer-software}

\emph{All source code publicly available at
\url{https://github.com/brian-rose/}}
\vspace{0.2 cm}

\years{2021 -- } \href{https://foundations.projectpythia.org}{Project Pythia Foundations}, a community learning resource for Python-based computing in the geosciences

\years{2014 -- } \href{https://github.com/brian-rose/climlab}{CLIMLAB}, a Python-based toolkit for interactive, process-oriented climate modeling

\years{2020 -- } \href{https://brian-rose.github.io/ClimateLaboratoryBook}{The Climate Laboratory}, an open-access interactive textbook

\years{2015 -- 2019} \href{https://github.com/brian-rose/ClimateModeling_courseware}{Climate Modeling Courseware}, interactive lecture notes in Jupyter notebook format

\years{2015 -- } \href{http://climlab.readthedocs.io/en/latest/}{CLIMLAB documentation}, online user manual for the CLIMLAB software package

\years{2015 -- } \href{https://github.com/brian-rose/pyCESM}{pyCESM}, Python-based analysis package for the Community Earth System Model


\section*{GRANT FUNDING}\label{grant-funding}

\subsection*{Federal}

\years{2020-2023} Collaborative Research: EarthCube Data Capabilities: Project Pythia: A Community Learning Resource for Geoscientists (PI, UAlbany component). NSF EarthCube, \$379,311. \emph{Status: funded, start date 9/1/2020}

\years{2020-2023} Dynamic and thermodynamic mechanisms of desert amplification in a warming climate (co-PI with Dr. Liming Zhou). NSF Climate and Large Scale Dynamics, \$696,071. \emph{Status: funded, start date 9/1/2020}

\years{2015 -- 2020} CAREER: Understanding the role of oceans in the planetary energy budget (PI). NSF Climate and Large Scale Dynamics, \$544,681. \emph{Status: ongoing}
  
\subsection*{University at Albany}

\years{2018} Does the Earth System have multiple stable states? (PI). FRAP-B award, \$2,000. \emph{Status: awarded}

\subsection*{Proposals not funded}

\years{2018-2023} Collaborative Research: Framework: Software: Community Earth System Informatics: Enabling Convergent Science (co-PI). NSF Cyberinfrastructure for Sustained Scientific Innovation, \$4,808,504. Lead PI is Matthew Long, National Center for Atmospheric Research. UAlbany component \$500,000. \emph{Status: not funded}
\vspace{0.2 cm}

\subsection*{In preparation}

%\years{2019-2021} State dependence of the Earth System response to orbital forcing (co-PI with Dr. Oliver Elison Timm). NSF Paleo Perspectives on Climate Change. \emph{To be submitted October 2018}.

\years{2021-2024} Understanding the vertical structure of Arctic climate change (sole PI). NSF Climate and Large Scale Dynamics. 


\section*{PRESENTATIONS}\label{presentations}

\subsection*{Invited Presentations}\label{invited-presentations}

\years{2021/06} Pangeo Showcase: Project Pythia: a community learning resource for Python-based computing in the geosciences.

\years{2021/02} Penn State University, Meteorology \& Atmospheric Sci.: Linking climate feedbacks to ocean heat uptake.

\years{2020/11} Colorado State University, Atmospheric Sci.: The efficiency of poleward heat transport into the Arctic.

\years{2019/09} McGill University, Atmospheric \& Oceanic Sci.: Why does climate sensitivity go up as ocean heat uptake declines? A linear systems perspective.

\years{2019/05} SIAM Conference on Applications of Dynamical Systems, minisymposium ``Planetary Motion and its Effects on Climate".

\years{2018/06} ICTP Summer School on Theory, Mechanisms and Hierarchical Modelling of Climate Dynamics: Multiple Equilibria in the Climate System.

\years{2018/06} Rossbypalooza, U. Chicago: Ice Caps and Ice Belts / Intro to CLIMLAB

\years{2018/01} AMS 17th Annual Student Conference, Tools of the Trade session: The Jupyter notebook.

\years{2017/09} Columbia University, SEAS Colloquium in Climate Science: Why does climate sensitivity go up as ocean heat uptake declines? A linear systems perspective.

\years{2017/05} York University, Earth \& Space Sci. \& Eng.: Global climate sensitivity goes up as ocean heat uptake declines: a linear systems perspective on inconstant climate feedbacks.

\years{2017/03} MIT, PAOC seminar: Why does climate sensitivity go up as ocean heat uptake declines? A linear systems perspective.

\years{2016/04} UW, Atmos. Sci.: The vertical structure of tropospheric water vapor: comparing radiative and ocean-driven climate changes.

\years{2016/04}  UW, Atmos. Sci.: Climate in the absence of ocean heat transport.

\years{2015/11} Columbia University, SEAS Colloquium in Climate Science: Understanding the effects of ocean circulation on radiative feedbacks and the planetary energy budget.

\years{2015/09} Stony Brook University, Marine \& Atmos. Sci.: Understanding the effects of ocean circulation on radiative feedbacks and the planetary energy budget.
  
\years{2015/01} Massachusetts College of Liberal Arts: What sets the temperature of the Earth? (public lecture) 
  
\years{2013/10} Caltech ESE seminar: The role of oceans in climate sensitivity and radiative feedbacks
  
\years{2013/10} Courant Institute, NYU: The role of oceans in climate sensitivity and radiative feedbacks.

\years{2013/05} SIAM Dynamical Systems conference: Multiple sea ice states and hysteresis in climate models.
  
\years{2013/03}  McGill University, Atmos. \& Oceanic Sci.: One wet planet, many climates.
  
\years{2013/03} UW, Atmos. Sci.: Climate sensitivity and the oceans.

\years{2013/01} U. Albany, Atmos. \& Environ. Sci.: One wet planet, many climates.
 
\years{2012/11} UW, Atmos. Sci.: Understanding why ocean heat transport matters: a multi-model approach.
  
\years{2012/05} MIT EAPS: Why does the climate system care about ocean heat transport?
   
\years{2012/04} UW, Oceanography: Modeling* the role of oceans and sea ice in multiple equilibria, abrupt climate change, and Snowball Earth (* and maybe understanding).
  
\years{2012/04}  U. Chicago, Geophysical Sci.: Water, water everywhere: role of oceans in warm climates.
  
\years{2012/03}  LDEO, Columbia U.: Why does the climate system care about ocean heat transport?
  
\years{2011/10}  U. Chicago, Geophysical Sci.: Why does the climate system care about ocean heat transport?
  
\years{2011/10}  UW, Oceanography: Why does the climate system care about ocean heat transport?
  
\years{2011/09}  ACDC2011, Friday Harbor WA: Ocean heat transport and weak temperature gradients. 
  
\years{2011/02}  CalTech, Environ. Sci. \& Eng.: Impact of ocean heat transport in cold and warm climates. 
  
\years{2011/01}  UW, Atmos. Sci.: Oceanic control of the sea ice edge and multiple equilibria.
  
\years{2010/09}  Harvard U., Earth and Planetary Sci.: Multiple equilibria of sea ice and climate.


\subsection*{Contributed conference presentations}\label{contributed-conference-presentations}

* \emph{indicates student co-author}
\vspace{0.2 cm}

\years{2021/06}  A. Banihirwe, D. Camron, J. Clyne, O. Eroglu, M. Grover, J. Kent, A. Kootz, M. Long, R. May, K. Paul, \underline{B.E.J. Rose}, M. Sizemore, K. Tyle, and A. Zacharias, Project Pythia: A Community Learning Resource for Geoscientists. EarthCube Annual Meeting 2021.

\years{2021/01} \underline{B.E.J. Rose}, M.C. Rencurrel*, M. Rugenstein, and K. Armour, Effects of spatial patterns of ocean heat uptake on Transient Climate Response (e-poster), AMS 34th Conference on Climate Variability and Change

\years{2021/01} \underline{B.E.J. Rose} and C. Cardinale*, The efficiency of poleward heat transport into the Arctic (oral presentation), AMS 34th Conference on Climate Variability and Change

\years{2020/12} Y.-J. Lin*, Y.-T. Hwang, J. Lu, F. Liu, and \underline{B.E.J. Rose}, Attributing Radiative Feedback Evolution to Regional Ocean Heat Uptake (oral presentation), AGU Fall Meeting

\years{2020/12} M.C. Rencurrel* and \underline{B.E.J. Rose}, Effects of Spatial Patterns of Ocean Heat Uptake on the CMIP5 inter-model spread of the Transient Climate Response (e-poster), AGU Fall Meeting

\years{2020/01} \underline{B.E.J. Rose}, CLIMLAB 2.0: Lessons Learned and Future Roadmap for Interactive, Process-Oriented Climate Modeling (oral presentation), AMS 10th Symposium on Advances in Modeling and Analysis Using Python.

\years{2020/01} \underline{B.E.J. Rose} and F. Zhu*, Multiple Equilibria in a Fully Coupled Carbon?Climate Model (poster), AMS Robert Dickinson Symposium and 33rd Conference on Climate Variability and Change.

\years{2020/01} \underline{B.E.J. Rose}, M.C. Rencurrel*, and M. Rugenstein, Effects of spatial patterns of ocean heat uptake on the inter-model spread of the Transient Climate Response (oral presentation), AMS 33rd Conference on Climate Variability and Change.

\years{2019/12} \underline{B.E.J. Rose}, Interactive Climate Modeling and Reproducible Workflows in the Classroom (oral presentation), AGU Fall Meeting

\years{2019/12} \underline{B.E.J. Rose} and M.C. Rencurrel*, Ocean heat transport makes the world warmer: coupled cloud-convection-circulation response of an aquaplanet to idealized surface forcing (poster), AGU Fall Meeting

\years{2019/12} Zhu, F.* and \underline{B.E.J. Rose}, Multiple Equilibria in a Fully Coupled Carbon-Climate Model (eLightning presentation), AGU Fall Meeting.

\years{2019/12} Cardinale, C.* and \underline{B.E.J. Rose}, Stratospheric and Tropospheric contributions to the Poleward Energy Flux across 70�N (poster), AGU Fall Meeting.

\years{2019/12} Rencurrel, M.C.*, \underline{B.E.J. Rose} and M. Rugenstein, Effects of spatial patterns of ocean heat uptake on the inter-model spread of the Transient Climate Response (poster), AGU Fall Meeting.

\years{2019/06} \underline{Rose, B.E.J.}, The Vertical Structure of Arctic Climate Change: a Single-Column Model Perspective (oral presentation), AMS Conference on Atmospheric and Oceanic Fluid Dynamics.

\years{2018/12} Rencurrel, M.C.* and \underline{B.E.J. Rose}, The Efficiency of the Hadley Cell Response to Wide Variations in Ocean Heat Transport (oral presentation), AGU Fall Meeting

\years{2018/10} Rencurrel, M.C.* and \underline{B.E.J. Rose}, The Efficiency of the Hadley Cell Response to Wide Variations in Ocean Heat Transport (poster), Understanding and Modeling the Earth's Climate, a symposium in honor of Isaac Held. Princeton University.

\years{2018/07} Rencurrel, M.C.* and \underline{B.E.J. Rose}, Exploring the Robust Hadley Cell Response to Variations in Ocean Heat Transport (poster), WCRP Grand Challenge on Clouds, Circulation and Climate Sensitivity: 2nd Meeting on Monsoons and Tropical Rain Belts.

\years{2018/01} \underline{Rose, B.E.J.} and C. Cardinale*, Stratospheric and Tropospheric Contributions to the Flux of Moist Static Energy across 70$^\circ$N (oral presentation), AMS 31st Conference on Climate Variability and Change.

\years{2018/01} \underline{Rose, B.E.J.}, A Computational Approach to Climate Science Education with CLIMLAB (oral presentation), AMS Eighth Symposium on Advances in Modeling and Analysis Using Python.

\years{2017/12} \underline{Rose, B.E.J.}, Climate in the absence of ocean heat transport (oral presentation), AGU Fall Meeting

\years{2017/12} \underline{Rose, B.E.J.}, A computational approach to climate science education with CLIMLAB (poster), AGU Fall Meeting

\years{2017/12} Rencurrel, M.C.* and \underline{B.E.J. Rose}, Understanding the robustness of Hadley cell response to wide variations in ocean heat transport (oral presentation), AGU Fall Meeting

\years{2017/12} Cardinale, C.* and \underline{B.E.J. Rose}, Stratospheric and Tropospheric Contributions to the Flux of Moist Static Energy Across 70$^\circ$N and 65$^\circ$S (poster), AGU Fall Meeting

\years{2017/06} \underline{Rose, B.E.J.}, T.W. Cronin and C.M. Bitz, Ice Caps and Ice Belts: the effects of obliquity on albedo feedback (oral presentation), AMS Conference on Atmospheric and Oceanic Fluid Dynamics.

\years{2017/06} Singh, H.A., P.J. Rasch and \underline{B.E.J. Rose}, Impact of Ocean Dynamics on Polar Climate Change (oral presentation), AMS Conference on Atmospheric and Oceanic Fluid Dynamics.

\years{2017/01} \underline{Rose, B.E.J.}, CLIMLAB: a Python-Based Software Toolkit for Interactive, Process-Oriented Climate Modeling, AMS Seventh Symposium on Advances in Modeling and Analysis Using Python.
  
\years{2016/12} \underline{Rose, B.E.J.} and L. Rayborn*, Climate sensitivity increases as ocean heat uptake declines: a linear systems perspective (oral presentation), AGU Fall Meeting.
  
\years{2016/12} \underline{Rose, B.E.J.}, Interactive, process-oriented climate modeling with CLIMLAB (oral presentation), AGU Fall Meeting.
  
\years{2016/12} Rayborn, L.* and \underline{B.E.J. Rose}, Understanding the Dependence of Radiative Feedbacks and Clouds on the Spatial Structure of Ocean Heat Uptake (oral presentation), AGU Fall Meeting.
  
\years{2016/12} Rencurrel, M.C.* and \underline{B.E.J. Rose}, Understanding Atmospheric Adjustment to Variations in Tropical Ocean Heat Transport (poster), AGU Fall Meeting.

\years{2016/11} \underline{Rose, B.E.J.}, CLIMLAB: a Python toolkit for interactive, process-oriented climate modeling (oral presentation), AOSPY workshop, Columbia University.
  
\years{2016/11} \underline{Rose, B.E.J.}, Robust non-local effects of ocean heat uptake on radiative feedback and subtropical cloud cover (oral presentation), Model Hierarchies workshop, Princeton.
  
\years{2016/02} \underline{Rose, B.E.J.}, Robust non-local effects of ocean heat uptake on radiative feedback and subtropical cloud cover (oral presentation), Ocean Sciences.
  
\years{2015/12} Rayborn, L.* and \underline{B.E.J. Rose}, Robust effects of ocean heat uptake on radiative feedback and subtropical cloud cover: a study using radiative kernels (oral presentation), AGU Fall Meeting.
  
\years{2015/12} Rencurrel, M.C.* and \underline{B.E.J. Rose}, Atmospheric compensation of variations in tropical ocean heat transport: understanding mechanisms and implications on tectonic timescales (poster), AGU Fall Meeting.
  
\years{2015/12} \underline{Rose, B.E.J.}, Climate in the absence of ocean heat transport (poster), AGU Fall Meeting.
  
\years{2015/12} \underline{Rose, B.E.J.}, CLIMLAB: a Python-based software toolkit for interactive, process-oriented climate modeling (poster), AGU Fall Meeting.
  
\years{2014/12} \underline{Rose, B.E.J.}, Accidental Lessons on Nonlinear Wind - Ocean - Sea Ice Interaction in the Tropics, with Implications for Snowball Earth (poster), AGU Fall Meeting.
  
\years{2014/06} \underline{Rose, B.E.J.}, The dependence of transient climate sensitivity and radiative feedbacks on the spatial pattern of ocean heat uptake (oral presentation), Latsis Symposium, ETH Zurich.
  
\years{2013/12} \underline{Rose, B.E.J.}, D. Battisti and K. Armour, The dependence of transient climate sensitivity and radiative feedbacks on the spatial pattern of ocean heat uptake (oral presentation), AGU Fall Meeting.
  
\years{2012/12} \underline{Rose, B.E.J.}, Understanding the atmospheric response to ocean heat transport: a model inter-comparison (oral presentation), AGU Fall Meeting.
  
\years{2011/12} \underline{Rose, B.E.J.}, D. Ferreira and J. Marshall, Not all poleward heat transport is created equal: a new look at warm climates, water vapor feedback, and the low-temperature-gradient paradox (oral presentation), AGU Fall Meeting.
  
\years{2011/06} \underline{Rose, B.E.J.}, D. Ferreira and J. Marshall, On the dynamics of an abrupt climate change (oral presentation), CMOS Congress, Victoria BC.
  
\years{2011/05} \underline{Rose, B.E.J.}, D. Ferreira and J. Marshall, On the dynamics of an abrupt climate change (oral presentation), AMS Polar Meteorology and Oceanography Conference, Boston MA.
  
\years{2010/07} \underline{Rose, B.E.J.}, Oceanic control of the sea ice edge and multiple equilibria in the climate system (thesis defense), MIT, Cambridge MA.
  
\years{2010/06} \underline{Rose, B.E.J.}, D. Ferreira and J. Marshall, Multiple equilibria and abrupt climate change in coupled Aquaplanet simulations (oral presentation), CMOS Congress, Ottawa ON.
  
\years{2010/05} \underline{Rose, B.E.J.}, Ocean heat transport, sea ice, and multiple equilibria of the climate system, Sack Lunch Seminar in Oceanography and Climate, MIT, Cambridge MA.
  
\years{2009/11} \underline{Rose, B.E.J.}, D. Ferreira and J. Marshall, Multiple equilibria of the atmosphere-ocean-ice system (oral presentation), Ocean-Atmosphere Energy Transport conference, CalTech, Pasadena CA.
  
\years{2009/04} \underline{Rose, B.E.J.}, Multiple equilibria of the atmosphere-ocean-ice system (oral presentation), Graduate Climate Conference, UW, Pack Forest WA.
  
\years{2008/04}
  \underline{Rose, B.E.J.} and J. Marshall, Heat transport, wind stress and the ice edge: new insights from simple models (oral presentation), CMOS Congress, Kelowna BC.
  
\years{2007/10} \underline{Rose, B.E.J.}, Sea ice, wind, and ocean currents: feedbacks and instabilities in ice age climates (oral presentation), Graduate Climate Conference, UW, Pack Forest WA.
  
\years{2007/05} \underline{Rose, B.E.J.} and J. Marshall, Constraints on atmospheric and oceanic heat transport from an idealized coupled climate model with sea-ice (oral presentation), CMOS-CGU-AMS Joint Congress, St.~John's NF.
  
\years{2006/04} \underline{Rose, B.E.J.}, The partition of heat transport in a simple coupled climate model (oral presentation), Graduate Climate Conference, UW, Pack Forest WA
  
\years{2004/06} \underline{Rose, B.E.J.} and C.A. Lin, A reconstruction of historical summer drought in Quebec based on tree rings (poster), Symposium Ouranos sur les changements climatiques, Montreal QC
  
\years{2001/02} \underline{Rose, B.E.J.} and C.A. Lin, Statistical relation between precipitation and vertical motion (oral presentation), Canadian CLIVAR Network Workshop, Victoria BC.


\section*{TEACHING AND ADVISING}\label{teaching-and-advising}

\subsection*{Courses taught at UAlbany}\label{courses}

* \emph{indicates newly developed courses}

\emph{Course websites at links below or from \url{http://www.atmos.albany.edu/facstaff/brose/}}

\vspace{0.2 cm}

\years{2021} A ATM 100 The Atmosphere

\years{2015--2019} \href{http://www.atmos.albany.edu/facstaff/brose/classes/ATM623_Spring2019/}{A ATM 623 Climate Modeling*}
  
\years{2015--2021} \href{http://www.atmos.albany.edu/facstaff/brose/classes/ATM500_Fall2019/}{A ATM 500 Atmospheric Dynamics*}
  
\years{2014--2020} \href{http://www.atmos.albany.edu/facstaff/brose/classes/ENV415_Spring2020/}{A ATM/ENV 415 Climate Laboratory*} (previously A ENV 480)
  
\years{2014} \href{http://www.atmos.albany.edu/facstaff/brose/classes/ATM316_Fall2014/}{A ATM 316 Dynamic Meteorology I} 
  
\years{2013} \href{http://www.atmos.albany.edu/facstaff/brose/classes/ATM619_Fall2013/}{A ATM 619 Oceans and Climate Seminar*}
  
  
\subsection*{Previous teaching}\label{old-courses}

\years{2013} UW ATMS 542 Geophysical Fluid Dynamics II, co-taught with D.S. Battisti
  
\years{2011, 2013}
  UW ATMS 514 / ESS 535 Ice and Climate, guest lectures for C.M. Bitz
  
\years{2011} Lecturer, ACDC2011, ``Dynamics of Past Warm Climates''
  
\years{2009} Lecture note preparation, P. O'Gorman, General Circulation of the Atmosphere, MIT
  
\years{2007} TA, guest lecturer, J. Marshall, Physics of Atmospheres and Oceans, MIT
  
\years{2006 -- 2007} Lab assistant, Fayerweather Street School, Cambridge MA
  
\years{2006} TA for R.S. Lindzen, Strange bedfellows: science and environmental policy, MIT


\subsection*{Graduate Students Advised}\label{graduate-students-advised}

\subsubsection*{Current}

\years{2021 -- } Yuan-Jen Lin (visiting PhD student, National Taiwan University)

\years{2021 -- } Robert Ford (advisor)

\years{2018 -- } Fangze Zhu (advisor, PhD qualifying exam 5/2019)

\years{2016 -- } Christopher Cardinale (advisor, MS completed 12/2018, PhD prospectus 7/2020)

\years{2021 -- } Alejandro Ayala (co-advisor, MS expected 2021)

\years{2016 -- } Lanxi Min (committee member, PhD prospectus 6/2019)

\years{2018 -- } Yan Jiang (committee member, PhD prospectus 9/2019)

\years{2019 -- }  Brendan Wallace (committee member, PhD prospectus 5/2020)

\years{2019 -- } Zhaoxiangrui He (committee member, PhD qualifying exam 5/2019)

\years{2019 -- }  Heather Sussman (committee member, PhD prospectus 11/2020)

\years{2020 -- } Kathrin Alber (committee member, PhD qualifying exam 12/2020)

\years{2020 -- } Rebecca Orrison (committee member, PhD qualifying exam 12/2020)

\years{2021 -- } Katrina Fandrich (committee member, PhD qualifying exam 5/2021)

\years{2021 -- } Matthew Jenkins (committee member, PhD qualifying exam anticipated 8/2021)

\subsubsection*{Completed}

\years{2014 -- 2020} Michael Cameron Rencurrel (advisor, MS completed 2/2017, PhD defended 4/2020)

\years{2015 -- 2016} Lance Rayborn (advisor, MS completed 12/2016)

\years{2018 -- 2020} Ajay Raghavendra (committee member, PhD defended 9/2020)

\years{2018 -- 2020} Hing Ong aka Heng Wang (committee member, PhD defended 4/2020)

\years{2015 -- 2019} Anthony Joyce (committee member, U. Massachusetts Amherst, PhD defended 6/2019)

\years{2016 -- 2019} Di Chen (committee member, PhD defended 5/2019)

\years{2014 -- 2018} Hannah Attard (committee member, PhD defended 4/2018)

\years{2013 -- 2017} Pablo Paiewonsky (committee member, PhD defended 6/2017)

\years{2017} Christine Bloecker (MS thesis reader, 5/2017)

\years{2014 -- 2017} Theodore Letcher (committee member, PhD defended 2/2017)

\years{2013 -- 2016} Christopher Colose (committee member, PhD defended 12/2016)

\years{2015} Melissa Gervais (external PhD thesis examiner, McGill University)


\subsection*{Undergraduate Students Advised}\label{undergraduate-students-advised}

\years{2019} Duan-Heng Chang (summer research intern, National Taiwan University)

\years{2015} Chyi-Rong Tsai (summer research intern, National Taiwan University)

\years{2014} Deborah McGlynn (senior thesis in Environmental Science)

\years{2013 --} Academic advisor for roughly 25 students in ATM and ENV majors


\section*{SERVICE}\label{service}

\subsection*{Departmental}\label{departmental}

\years{2021 -- } Faculty mentor to Assistant Prof. Sujata Murty

\years{2020 -- } Faculty lead on development of Climate Science graduate degree programs

\years{2019 -- 2020} Faculty search committee in Earth System Science (resulted in two tenure-track hires: Dr. Sujata Murty and Dr. Aubrey Hillman)

\years{2019 -- } E-TEC building committee, member

\years{2018 -- 2019} Faculty organizer for the DAES graduate student recruitment visit

\years{2017} Represented DAES at DEC Pack Forest camp College Exploration event

\years{2015 -- } DAES graduate committee member

\years{2014 -- } Chair, planning committee for GFD / Env. Sci. teaching laboratory in
  E-TEC building

\years{2014 -- } Organizer, DAES Climate Group weekly seminar series

\years{2015 -- } Transfer student advising


\subsection*{College of Arts and Sciences}\label{college-of-arts-and-sciences}

\years{2016 -- 2019} CAS Faculty Council (at-large councillor)

\years{2016 -- 2017} CAS Academic Planning Committee (inactive)

\years{2017 -- 2019} CAS Academic Support Committee


\subsection*{University at Albany}\label{university}

\years{2021 -- 2022} Council on Research (chair)

\years{2019 -- 2021} Council on Research (member, chair of ERCA subcommittee, member of FRAP-A review subcommittee)

\years{2019 -- } University Senate (department representative)

\years{2018 -- } Udall Scholarship review committee

\years{2017} Strategic Planning Steering Committee 

%\years{2018 -- 2019} CAS liason to University Senate LISC committee on SUNY Open Access initiative


\subsection*{Professional}\label{professional}

\years{2021} Session co-chair (multiple) \& student judge, AMS 34th Conference on Climate Variability and Change

\years{2020 -- 2021} Program chair, AMS 34th Conference on Climate Variability and Change

\years{2019 -- 2022} Member of AMS Climate Variability and Change Committee 

\years{2018/05} Panel review member for DOE Regional and Global Model Analysis program

\years{2017 -- 2018} Member of Advisory Committee, 2018 CESM Polar Modeling Workshop.

\years{2009 -- }  Reviewer for Nature, J. Climate, J. Atmos. Sci., J. Geophys. Res.,
  Geophys. Res. Lett., Nature Geosci., Nature Clim. Change, Nature Comm., JAMES, 
  Climate Dynamics, Astrophys. J., Mon. Not. R. Astron. Soc., SIAM J.
  Appl. Dyn. Sys., Earth Sys. Dyn., \& Encyclopedia of Natural Resources

\years{2015 -- }
  Proposal reviewer for NSF and Israel Science Foundation

\years{2015}
  Session Convener: ``Polar Climate and Predictability'', AGU Fall Meeting.
  
\years{2014}  Session Convener: ``Innovative Insights into the Climate System and Climate Models: Exploring Scales and Parameter Spaces'', AGU Fall Meeting.
  
\years{2013}
  Judge for Outstanding Student Presentation Awards, AGU Fall Meeting.
  
\years{2012/07}
  Convener, Workshop on heat transport in aquaplanet models, UW Atmos. Sci..
  
\years{2011/04}
  Moderator, NOAA C\&GC Postdoctoral Program 20th anniversary celebration.
  
\years{2009/04}
  Chair (invited), ocean circulation session, 3rd Graduate Climate Conference, UW.


\subsection*{Community}\label{community}

\years{2018 -- 2020} UAlbany Family Earth Day, lead faculty organizer

\years{2018/10} Public seminar: ``Climate Sensitivity in an Uncertain World", Science on Tap.

\years{2016 -- 2018} UAlbany Family Earth Day, ``weather in a tank" demonstrations

\years{2014/07} Space Science and Next Generation of Science Standards (forum for high school science teachers), lecture on climate change and climate modeling, RPI.

\years{2007 -- 2009}
  Session leader, YouthCAN Summit on Global Warming, MIT.
  
 \years{2008/01}
  Public seminar: ``Looking Back on the Future of Climate Change'', MIT.


\section*{HONORS AND AWARDS}\label{honors-and-awards}

\years{2020}
2019 Editors' Citation for Excellence in Refereeing for \emph{Geophysical Research Letters}

\years{2012}
  Commendation for exceptional refereeing, Nature Publishing Group.

\years{2010 -- 2012}
  NOAA Climate and Global Change Postdoctoral Fellowship
  
\years{2010}
  Carl-Gustav Rossby Prize for best thesis, MIT
  
\years{2004}
  Jule G. Charney Prize and MIT Presidential Fellowship
  
\years{2002}
  Dean's Honour List for M.Sc. thesis, McGill University
  
\years{2001 -- 2002}
  NSERC Graduate Fellowship, McGill University

\years{2001}
  Meteorological Service of Canada supplement to NSERC Fellowship (declined)
  
\years{1999}
  NSERC Undergraduate Research Fellowship
  
\years{1995 -- 1999}
  James McGill Scholarship and J.S. Marshall Prize, McGill University
  

\section*{SUMMER SCHOOLS AND WORKSHOPS}\label{summer-schools}

\years{2019/10} CMIP6 Hackathon, LDEO, Columbia University.

\years{2018/07} WCRP Grand Challenge on Clouds, Circulation and Climate Sensitivity: 2nd Meeting on Monsoons and Tropical Rain Belts, ICTP, Trieste, Italy.

\years{2018/06} ICTP Summer School on Theory, Mechanisms and Hierarchical Modelling of Climate Dynamics: Multiple Equilibria in the Climate System, Trieste, Italy.

\years{2018/06} Rossbypalooza, ``Understanding climate through simple models", U. Chicago.

\years{2016/11} AOSPY / Pangeo scientific software workshop, Columbia University.

\years{2016/11} Model Hierarchies Workshop, Princeton University.

\years{2015/09} Monsoons and the ITCZ workshop, Columbia University.

\years{2012/09} PCC Summer Institute: Atmosphere-Ocean-Ice Shelf Interactions, Friday Harbor, WA. 

\years{2012/07} Workshop on heat transport in aquaplanet models, University of Washington.

\years{2012/07} NOAA Climate and Global Change Summer Institute, Steamboat Springs, CO.

\years{2011/09} ACDC2011:  Dynamics of Past Warm Climates, Friday Harbor, WA.

\years{2009/05} Fundamental Problems in Climate Dynamics, Princeton University.

\years{2007/07} International Sea Ice Summer School, Svalbard.


\section*{PROFESSIONAL AFFILIATIONS}\label{professional-affiliations}

  American Geophysical Union

  American Meteorological Society




%\vfill{} % Whitespace before final footer

%----------------------------------------------------------------------------------------
%	FINAL FOOTER
%----------------------------------------------------------------------------------------

%\begin{center}
%{\scriptsize Last updated: \today\- •\- \href{http://www.LaTeXTemplates.com}{http://www.LaTeXTemplates.com}} % Any final footer text such as a URL to the latest version of your CV, last updated date, compiled in XeTeX, etc
%{\scriptsize Last updated: \today}
%\end{center}

%----------------------------------------------------------------------------------------

\end{document}